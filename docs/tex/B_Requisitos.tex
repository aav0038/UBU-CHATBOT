\apendice{Especificación de Requisitos}

\section{Introducción}

En este apartado se detallan los objetivos generales del proyecto, los requisitos y la especificación de los mismos.  
Se analizan tanto los requisitos funcionales como los no funcionales. 
\begin{itemize}
	\tightlist
	\item
	Requisitos funcionales: comportamientos específicos que debe tener el sistema. Se relacionan con los casos de uso.
	\item
	Requisitos no funcionales: especifican los criterios a seguir, restricciones y condiciones que impone el cliente.

\end{itemize}

\section{Objetivos generales}


\begin{itemize}
	\tightlist
	\item
	Implementar un \textit{chatbot} que responda las preguntas frecuentes de la asignatura Trabajo de Fin de Grado tanto para la modalidad online como presencial.
	\item
	Ser capaces de dar respuesta al mayor número de preguntas posible.
	\item
	Integrar el \textit{chatbot} en UBUVirtual y Slack.
	\item
	Diseñar una interfaz conversacional que resulte agradable a los usuarios y respete la imagen corporativa.
	\item
	Capturar los \textit{logs} de las conversaciones para analizarlos y extraer resultados. 
\end{itemize}
\newpage

\section{Catalogo de requisitos}


\subsection{Requisitos funcionales}\label{requisitosFuncionales}

\begin{itemize}
	\tightlist
	\item
	\textbf{RF-1 Interacción texual:} la aplicación debe permitir al usuario interactuar con ella mediante texto.
	\item
	\textbf{RF-2 Reconocimiento de preguntas:} la aplicación debe ser capaz de reconocer las preguntas que introduce el usuario.
	\item
	\textbf{RF-3 Respuesta a las preguntas:} la aplicación debe ser capaz de responder a las preguntas que le formule el usuario.
	\item
	\textbf{RF-4 Información mediante hipervínculos:} la aplicación debe adjuntar como hipervínculos las redirecciones a otras páginas.
	\item
	\textbf{RF-5 Minimizar la interfaz conversacional:} la aplicación debe permitir minimizar la interfaz conversacional.
	\item
	\textbf{RF-6 Expandir la interfaz conversacional:} la aplicación debe permitir expandir la interfaz conversacional.
	\item
	\textbf{RF-7 Informe de error:} la aplicación debe informar de un error al usuario cuando su mensaje de entrada no sea un texto válido o no haya sido posible interpretarlo. 
\end{itemize}


\subsection{Requisitos no funcionales}\label{requisitosNoFuncionales}

\begin{itemize}
	\tightlist
	\item
	\textbf{RNF-1 Rendimiento:} la aplicación tiene que tener un tiempo de respuesta bajo.  
	\item
	\textbf{RNF-2 Usabilidad:} la aplicación debe ser intuitiva y fácil de entender y utilizar.  
	\item
	\textbf{RNF-3 Colores corporativos:} la aplicación debe mantener los colores corporativos de UBUVirtual.  
	\item
	\textbf{RNF-4 Disponibilidad:} la aplicación debe estar disponible el mayor tiempo posible.
	\item
	\textbf{RNF-5 Mantenibilidad:} la aplicación debe ser fácilmente modificable.
	\item
	\textbf{RNF-6 Registro de actividad:} la aplicación debe almacenar las interacciones llevadas a cabo con los usuarios.
	\item
	\textbf{RNF-7 Portabilidad:} la aplicación debe almacenar las interacciones llevadas a cabo con los usuarios.
	\item
	\textbf{RNF-8 Compatibilidad de navegadores:} la aplicación debe ser compatible para los navegadores más importantes, incluyendo también los navegadores de \textit{smartphones}.
\end{itemize}

\newpage
\section{Especificación de requisitos}

\subsection{Diagrama de casos de uso}
\imagen{diagramaCasosDeUso}{Diagrama de casos de uso.}

El actor -usuario- va a ser el estudiante, profesor o cualquier persona con acceso a la asignatura que comience una nueva sesión con el \textit{chatbot}.

\newpage

\subsection{Casos de uso}

\begin{longtable}[H]{@{}ll@{}}
	\toprule
	\begin{minipage}[b]{0.23\columnwidth}\raggedright\strut
		\textbf{CU-01}\strut
	\end{minipage} & \begin{minipage}[b]{0.71\columnwidth}\raggedright\strut
		\textbf{Expandir la interfaz conversacional.}\strut
	\end{minipage}\tabularnewline
	\midrule
	\endhead  
	\begin{minipage}[t]{0.23\columnwidth}\raggedright\strut
		\textbf{Requisitos asociados}\strut
	\end{minipage} & \begin{minipage}[t]{0.71\columnwidth}\raggedright\strut
		RF-6\strut
	\end{minipage}\tabularnewline
	\begin{minipage}[t]{0.23\columnwidth}\raggedright\strut
		\textbf{Descripción}\strut
	\end{minipage} & \begin{minipage}[t]{0.71\columnwidth}\raggedright\strut
		Permite al usuario expandir la interfaz conversacional del \textit{chatbot}.\strut
	\end{minipage}\tabularnewline
	\begin{minipage}[t]{0.23\columnwidth}\raggedright\strut
		\textbf{Precondición}\strut
	\end{minipage} & \begin{minipage}[t]{0.71\columnwidth}\raggedright\strut
		El usuario está logeado en UBUVirtual y dentro de la asignatura Trabajo Fin de Grado, en el apartado \textit{Chatbot de preguntas frecuentes}.\\
		El \textit{chatbot} esta minimizado.\strut
	\end{minipage}\tabularnewline
	\begin{minipage}[t]{0.23\columnwidth}\raggedright\strut
		\textbf{Acciones}\strut
	\end{minipage} & \begin{minipage}[t]{0.71\columnwidth}\raggedright\strut
		\begin{enumerate}
			\def\labelenumi{\arabic{enumi}.}
			\tightlist
			\item
			El usuario accede a la página en que aparece el icono del \textit{chatbot}.
			\item
			El usuario hace clic en el botón circular con el logo del proyecto.
			\item
			Se expande la interfaz conversacional.
		\end{enumerate}\strut
	\end{minipage}\tabularnewline
	\begin{minipage}[t]{0.23\columnwidth}\raggedright\strut
		\textbf{Postcondición}\strut
	\end{minipage} & \begin{minipage}[t]{0.71\columnwidth}\raggedright\strut
		La interfaz conversacional queda expandida y el usuario puede interactuar con el \textit{chatbot}.\strut
	\end{minipage}\tabularnewline
	\begin{minipage}[t]{0.23\columnwidth}\raggedright\strut
		\textbf{Excepciones}\strut
	\end{minipage} & \begin{minipage}[t]{0.71\columnwidth}\raggedright\strut
		-\strut
	\end{minipage}\tabularnewline
	\begin{minipage}[t]{0.23\columnwidth}\raggedright\strut
		\textbf{Importancia}\strut
	\end{minipage} & \begin{minipage}[t]{0.71\columnwidth}\raggedright\strut
		Alta\strut
	\end{minipage}\tabularnewline
	\bottomrule
	\caption{CU-01 Expandir la interfaz conversacional.}
\end{longtable}

\newpage
\begin{longtable}[H]{@{}ll@{}}
	\toprule
	\begin{minipage}[b]{0.23\columnwidth}\raggedright\strut
		\textbf{CU-02}\strut
	\end{minipage} & \begin{minipage}[b]{0.71\columnwidth}\raggedright\strut
		\textbf{Formular una pregunta.}\strut
	\end{minipage}\tabularnewline
	\midrule
	\endhead  
	\begin{minipage}[t]{0.23\columnwidth}\raggedright\strut
		\textbf{Requisitos asociados}\strut
	\end{minipage} & \begin{minipage}[t]{0.71\columnwidth}\raggedright\strut
		RF-1, RF-2, RF-3, RF-4, RF-7\strut
	\end{minipage}\tabularnewline
	\begin{minipage}[t]{0.23\columnwidth}\raggedright\strut
		\textbf{Descripción}\strut
	\end{minipage} & \begin{minipage}[t]{0.71\columnwidth}\raggedright\strut
		El usuario introduce de manera textual una pregunta al \textit{chatbot}.\strut
	\end{minipage}\tabularnewline
	\begin{minipage}[t]{0.23\columnwidth}\raggedright\strut
		\textbf{Precondición}\strut
	\end{minipage} & \begin{minipage}[t]{0.71\columnwidth}\raggedright\strut
		El usuario está logeado en UBUVirtual y dentro de la asignatura Trabajo Fin de Grado, en el apartado \textit{Chatbot de preguntas frecuentes}.\\
		El \textit{chatbot} esta expandido.\strut
	\end{minipage}\tabularnewline
	\begin{minipage}[t]{0.23\columnwidth}\raggedright\strut
		\textbf{Acciones}\strut
	\end{minipage} & \begin{minipage}[t]{0.71\columnwidth}\raggedright\strut
		\begin{enumerate}
			\def\labelenumi{\arabic{enumi}.}
			\tightlist
			\item
			El usuario escribe y envía por medio de la interfaz conversacional una pregunta al \textit{chatbot}.
			\item
			El programa analiza la pregunta introducida e intenta reconocerla.
			\item
			Se responde al usuario en modo texto y por medio de la interfaz conversacional a su pregunta. 
			\item
			El \textit{chatbot} queda a la espera de recibir nuevas preguntas.
		\end{enumerate}\strut
	\end{minipage}\tabularnewline
	\begin{minipage}[t]{0.23\columnwidth}\raggedright\strut
		\textbf{Postcondición}\strut
	\end{minipage} & \begin{minipage}[t]{0.71\columnwidth}\raggedright\strut
		Se devuelve un mensaje con la respuesta a la pregunta.\strut
	\end{minipage}\tabularnewline
	\begin{minipage}[t]{0.23\columnwidth}\raggedright\strut
		\textbf{Excepciones}\strut
	\end{minipage} & \begin{minipage}[t]{0.71\columnwidth}\raggedright\strut
		Si el \textit{chatbot} no es capaz de reconocer la pregunta introducida por el usuario o está mal formulada se informa al usuario por medio de un mensaje de que no ha sido posible entenderle.\strut
	\end{minipage}\tabularnewline
	\begin{minipage}[t]{0.23\columnwidth}\raggedright\strut
		\textbf{Importancia}\strut
	\end{minipage} & \begin{minipage}[t]{0.71\columnwidth}\raggedright\strut
		Alta\strut
	\end{minipage}\tabularnewline
	\bottomrule
	\caption{CU-02 Formular una pregunta.}
\end{longtable}

\newpage
\begin{longtable}[H]{@{}ll@{}}
	\toprule
	\begin{minipage}[b]{0.23\columnwidth}\raggedright\strut
		\textbf{CU-03}\strut
	\end{minipage} & \begin{minipage}[b]{0.71\columnwidth}\raggedright\strut
		\textbf{Formular una pregunta. (Slack) }\strut
	\end{minipage}\tabularnewline
	\midrule
	\endhead  
	\begin{minipage}[t]{0.23\columnwidth}\raggedright\strut
		\textbf{Requisitos asociados}\strut
	\end{minipage} & \begin{minipage}[t]{0.71\columnwidth}\raggedright\strut
		RF-1, RF-2, RF-3, RF-4, RF-7\strut
	\end{minipage}\tabularnewline
	\begin{minipage}[t]{0.23\columnwidth}\raggedright\strut
		\textbf{Descripción}\strut
	\end{minipage} & \begin{minipage}[t]{0.71\columnwidth}\raggedright\strut
		El usuario introduce de manera textual una pregunta al \textit{chatbot}.\strut
	\end{minipage}\tabularnewline
	\begin{minipage}[t]{0.23\columnwidth}\raggedright\strut
		\textbf{Precondición}\strut
	\end{minipage} & \begin{minipage}[t]{0.71\columnwidth}\raggedright\strut
		El usuario está logeado en el espacio de trabajo de Slack y en la aplicación \textit{UBU Asistante Virtual}.
	\end{minipage}\tabularnewline
	\begin{minipage}[t]{0.23\columnwidth}\raggedright\strut
		\textbf{Acciones}\strut
	\end{minipage} & \begin{minipage}[t]{0.71\columnwidth}\raggedright\strut
		\begin{enumerate}
			\def\labelenumi{\arabic{enumi}.}
			\tightlist
			\item
			El usuario escribe y envía por medio de la interfaz conversacional una pregunta al \textit{chatbot	}.
			\item
			El programa analiza la pregunta introducida e intenta reconocerla.
			\item
			Se responde al usuario en modo texto y por medio de la interfaz conversacional a su pregunta. 
			\item
			El \textit{chatbot} queda a la espera de recibir nuevas preguntas.
		\end{enumerate}\strut
	\end{minipage}\tabularnewline
	\begin{minipage}[t]{0.23\columnwidth}\raggedright\strut
		\textbf{Postcondición}\strut
	\end{minipage} & \begin{minipage}[t]{0.71\columnwidth}\raggedright\strut
		Se devuelve un mensaje con la respuesta a la pregunta.\strut
	\end{minipage}\tabularnewline
	\begin{minipage}[t]{0.23\columnwidth}\raggedright\strut
		\textbf{Excepciones}\strut
	\end{minipage} & \begin{minipage}[t]{0.71\columnwidth}\raggedright\strut
		Si el \textit{chatbot} no es capaz de reconocer la pregunta introducida por el usuario o está mal formulada se informa al usuario por medio de un mensaje de que no ha sido posible entenderle.\strut
	\end{minipage}\tabularnewline
	\begin{minipage}[t]{0.23\columnwidth}\raggedright\strut
		\textbf{Importancia}\strut
	\end{minipage} & \begin{minipage}[t]{0.71\columnwidth}\raggedright\strut
		Alta\strut
	\end{minipage}\tabularnewline
	\bottomrule
	\caption{CU-03 Formular una pregunta. (Slack)}
\end{longtable}

\newpage


\begin{longtable}[H]{@{}ll@{}}
	\toprule
	\begin{minipage}[b]{0.23\columnwidth}\raggedright\strut
		\textbf{CU-04}\strut
	\end{minipage} & \begin{minipage}[b]{0.71\columnwidth}\raggedright\strut
		\textbf{Ocultar interfaz conversacional.}\strut
	\end{minipage}\tabularnewline
	\midrule
	\endhead  
	\begin{minipage}[t]{0.23\columnwidth}\raggedright\strut
		\textbf{Requisitos asociados}\strut
	\end{minipage} & \begin{minipage}[t]{0.71\columnwidth}\raggedright\strut
		RF-5\strut
	\end{minipage}\tabularnewline
	\begin{minipage}[t]{0.23\columnwidth}\raggedright\strut
		\textbf{Descripción}\strut
	\end{minipage} & \begin{minipage}[t]{0.71\columnwidth}\raggedright\strut
		Permite al usuario ocultar la interfaz conversacional del \textit{chatbot}.\strut
	\end{minipage}\tabularnewline
	\begin{minipage}[t]{0.23\columnwidth}\raggedright\strut
		\textbf{Precondición}\strut
	\end{minipage} & \begin{minipage}[t]{0.71\columnwidth}\raggedright\strut
		El usuario está logeado en UBUVirtual y dentro de la asignatura Trabajo Fin de Grado, en el apartado \textit{Chatbot de preguntas frecuentes}.\\
		El \textit{chatbot} esta expandido.\strut
	\end{minipage}\tabularnewline
	\begin{minipage}[t]{0.23\columnwidth}\raggedright\strut
		\textbf{Acciones}\strut
	\end{minipage} & \begin{minipage}[t]{0.71\columnwidth}\raggedright\strut
		\begin{enumerate}
			\def\labelenumi{\arabic{enumi}.}
			\tightlist
			\item 
			El usuario hace clic en el botón circular con el logo del proyecto.
			\item
			La interfaz conversacional queda oculta.
		\end{enumerate}\strut
	\end{minipage}\tabularnewline
	\begin{minipage}[t]{0.23\columnwidth}\raggedright\strut
		\textbf{Postcondición}\strut
	\end{minipage} & \begin{minipage}[t]{0.71\columnwidth}\raggedright\strut
		La interfaz conversacional queda oculta.\strut
	\end{minipage}\tabularnewline
	\begin{minipage}[t]{0.23\columnwidth}\raggedright\strut
		\textbf{Excepciones}\strut
	\end{minipage} & \begin{minipage}[t]{0.71\columnwidth}\raggedright\strut
		-\strut
	\end{minipage}\tabularnewline
	\begin{minipage}[t]{0.23\columnwidth}\raggedright\strut
		\textbf{Importancia}\strut
	\end{minipage} & \begin{minipage}[t]{0.71\columnwidth}\raggedright\strut
		Alta\strut
	\end{minipage}\tabularnewline
	\bottomrule
	\caption{CU-04 Ocultar interfaz conversacional.}
\end{longtable}