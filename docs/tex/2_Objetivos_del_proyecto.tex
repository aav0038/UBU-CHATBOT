\capitulo{2}{Objetivos del proyecto}

Estos son los distintos objetivos con los que se ha realizado el proyecto.

\section{Objetivos generales}\label{objetivos-generales}

\begin{itemize}
	\tightlist
	\item
	Estudiar y evaluar las distintas plataformas y \textit{frameworks} que permiten trabajar con \emph{chatbots}, seleccionando la más adecuada de entre aquellas gratuitas y con soporte para castellano.
	\item
	Desarrollar un \emph{chatbot} que permita interaccionar con los estudiantes y darle respuestas a sus dudas con respecto a la asignatura del Trabajo de Fin de Grado.
	\item
	Integrar en la plataforma de UBUVirtual, bajo las restricciones dadas por la Universidad en cuanto a la edición del código fuente de la web.
	\item 
	Integrarlo también en otra plataforma.
	\item
	Implementar un sistema de \emph{logs} que permita monitorizar las interacciones con los usuarios.
	\item
	Explotar los datos obtenidos en los \emph{logs}, para aplicarles técnicas de \emph{machine learning}. 
\end{itemize}


\section{Objetivos técnicos}\label{objetivos-tecnicos}

\begin{itemize}
	\tightlist
	\item
	Integrar el \emph{chatbot} en la plataforma Dialogflow.
	\item
	Utilizar la herramienta Dialogflow Messenger para disponer de las respuestas avanzadas.
	\item 
	Utilizar las \emph{payload response} para dotar a nuestro chatbot de funcionalidades más avanzadas.
	\item
	Obtener el código HTML para su integración.
	\item
	Generar el código CSS que le de una apariencia estéticamente agradable al integrarlo en UBUVirtual.
	\item
	Generar un Moodle propio para testear la integración.
	\item
	Integrar el \emph{chatbot} en UBUVirtual.
	\item
	Implementar e integrar el \emph{chatbot} en Slack.
	\item
	Monitorizar las conversaciones de los usuarios.
	\item
	Aplicar técnicas de machine learning para los datos obtenidos en la monitorización.
	\item
	Utilizar el control de versiones mediante GitHub.
	
	
	
\end{itemize}