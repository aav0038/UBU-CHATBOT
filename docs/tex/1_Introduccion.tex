\capitulo{1}{Introducción}

\section{Contenido del trabajo}

Llegado al último curso de grado todos los alumnos de la UBU deben realizar su Trabajo de Fin de Grado (TFG) y son muchas las dudas que tienen acerca del funcionamiento del mismo, especialmente en la enseñanza online en la que al no haber un contacto tan estrecho con profesores y otros alumnos se tiene menos información. Motivo por el que los alumnos, aparte de consultar la extensa documentación aportada en las distintas plataformas de la UBU consultan a los profesores por email para despejar sus dudas.

Muchas de estas preguntas acerca del trabajo ya han sido respondidas anteriormente, o están explicadas en alguna parte de la documentación de UBUVirtual o de la página del grado. Con el propósito de ayudar a los alumnos a despejar estas dudas y reducir la carga de preguntas que reciben los profesores, se desarrolla este chatbot cuya finalidad es la de dar respuesta de forma automática a las preguntas de los usuarios acerca de todos los aspectos relacionados con la asignatura. Como restricción ha de utilizar únicamente herramientas gratuitas. El chatbot está integrado en la plataforma de UBUVirtual de la asignatura de TFG, a la que todos los alumnos matriculados en el trabajo tienen acceso. 

Por medio del aprendizaje automático nuestro chatbot será capaz de entender y responder las preguntas que los alumnos le formulen, y a su vez aprenderá de aquellas que no las haya recibido anteriormente para ser capaz de responderlas en la próxima ocasión en que se le pregunten.\\
Con ello resolveremos la doble problemática para profesores y alumnos, con la ventaja añadida de que el bot responderá ininterrumpidamente a cualquier hora del día.

\newpage

\section{Estructura de la memoria}

La memoria utiliza la siguiente estructura:

\begin{itemize}
	\tightlist
	\item
	\textbf{Introducción:} resumen del contenido del trabajo: problema, proceso y solución. Estructura de la memoria y listado de los materiales adjuntos.
	\item
	\textbf{Objetivos del proyecto:} listado de objetivos generales y técnicos del proyecto.
	\item
	\textbf{Conceptos teóricos:} explicación de los conceptos teóricos necesarios para comprender el proyecto.
	\item
	\textbf{Técnicas y herramientas:} listado de técnicas y herramientas analizadas para el desarrollo. Comparativa y justificación de la elección.
	\item
	\textbf{Aspectos relevantes del desarrollo del proyecto:} resumen de las distintas fases del desarrollo. Análisis de resultados obtenidos por el programa desarrollado.
	\item
	\textbf{Trabajos relacionados:} estado del arte en el campo de los \textit{chatbots} en la UBU y otras unióndades españolas.
	\item
	\textbf{Conclusiones y Líneas de trabajo futuras:} conclusiones extraídas de la elaboración del proyecto y pasos a seguir en la futura explotación y mejora.
\end{itemize}


La estructura de los anexos es la siguiente:

\begin{itemize}
	\tightlist
	\item
	\textbf{Plan de Proyecto Software:} planificación temporal y estudio de viabilidad económica y legal del proyecto.
	\item
	\textbf{Especificación de requisitos:} fase de análisis: objetivos generales, catálogo de requisitos y especificación de requisitos.
	\item
	\textbf{Especificación de diseño:} diseño de datos, diseño procedimental y diseño arquitectónico.
	\item
	\textbf{Documentación técnica de programación:} estructura de directorios, manual del programador, instalación del proyecto y pruebas del sistema.
	\item
	\textbf{Documentación de usuario:} manual para que el usuario sea capaz de utilizar de manera adecuada el \textit{chatbot} en sus distintas integraciones
\end{itemize}


\section{Materiales adjuntos}

Los materiales adjuntos al proyecto son los siguientes:

\begin{itemize}
	\tightlist
	\item
	Proyecto de Dialogflow para el \text{Chatbot} de la modalidad online.
	\item
	Proyecto de Dialogflow para el \text{Chatbot} para la modalidad presencial.
	\item
	Listado de preguntas y respuestas para la modalidad online.
	\item
	Listado de preguntas y respuestas para la modalidad presencial.
	\item
	Código para la integración HTML en UBUVirtual modalidad online.
	\item
	Código para la integración HTML en UBUVirtual modalidad presencial.

\end{itemize}

Están disponibles a través de Internet en el siguiente recurso:
\begin{itemize}
	\tightlist
	\item
	Repositorio del proyecto en Github \cite{repositorioGithub}.
\end{itemize}




